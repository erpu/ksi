\documentclass[10pt,twoside,a4paper]{article}

%\usepackage{abstract}
\usepackage{algpseudocode}
\usepackage{algorithm}
\usepackage[english]{babel}
\usepackage{amsmath}
	%\usepackage{apacite}
%\usepackage{apl}
	%\usepackage[polski]{babelbib}
	%\usepackage{bibtopic}
	%\usepackage[
	%backend=biber,
	%style=alphabetic,
	%sorting=ynt
	%]{biblatex}
\usepackage[utf8]{inputenc}
\usepackage[nottoc,numbib]{tocbibind}

\usepackage{cite}
\usepackage{latexsym}
\usepackage{indentfirst}
\usepackage{listings}
%\usepackage{graphicx} 
	%\usepackage[left=2cm,right=2cm,top=2cm,bottom=2cm]{geometry}
	%\usepackage{hyperref}
%\usepackage{indentfirst}

	%\usepackage{makeidx}
%\usepackage{natbib}

%\usepackage{setspace}
	%\usepackage{tabulary}
	%\usepackage[nottoc]{tocbibind}

	%\setlength{\parindent}{3pt}

\title{Web Page Classification Information System  as\\ an information software system\\ reducing access to Internet resources\\ using keywords with wages}
\author{Piotr Wojcik\\
    PJA,\\
    Warszawa,
    \texttt{piotr.wojcik@pja.edu.pl}}
\date{\today}


\begin{document}
%\frenchspacing
\maketitle
\bibliographystyle{plunsrt}
%\cleardoublepage

%\tableofcontents
%\clearpage

\section{Abstract}
Keywords:
\textit{Internet, classificator, classification, keywords with wage, DNS}

\section{Introduction}

\section{DNS}
	
\section{Keywords with wages}

\section{Keywords wages update and datamining}

%\lstinputlisting[language=c]
\lstdefinelanguage{csharp} {morekeywords={foreach, if, in, like}, sensitive=false, morecomment=[l]{//}, morecomment=[s]{/*}{*/}, morestring=[b]", }
\lstset{
numbers=left, numberstyle=\tiny, stepnumber=1, numbersep=5pt
}
\begin{lstlisting}[title={Alg 1. Podstawowy algorytm klasyfikacji stron internetowych}, language=csharp, label=alg1] 

//input: 
//    phrase - analysed phrase, 
//    analysedPage - web page with URL and other characteristics, 
//    compromisedWebPages - colection of compromised web pages, 
//    redFlagPhraseDictionary - colection of phrases 
//                              that compromise web page
//output:
//    score - points of page classification
//    compromisedWebPages - modyfied input object

mostCompromisedPages = compromisedWebPages.top();
foreach (phrase in webPageTextContent)
{  
  if((phrase in redFlagPhraseDictionary) 
      || (phrase like redFlagPhraseDictionary))
  {
    redFlagPhraseDictionary.computeWage(phrase, mostCompromisedPages);
    score += redFlagPhraseDictionary.wage(phrase);
    compromisedWebPages += analysedPage;
  }
}
\end{lstlisting}



\begin{enumerate}
\item 
\item 
\item 
\end{enumerate}

\section{Web pages category}

\begin{itemize}
\item 
\item 
\item 
\item 
\item 
\item 
\item 
\item 
\item 
\item 
\item 
\end{itemize}

\section{Web Page Classification Information System modules}

\section{Web Page Classification Information System appliance}

\section{Conclusion and future plans}



%\begin{figure}
%\centering
%\includegraphics[width=\textwidth]{eegsignal.png}
%\includegraphics{eegsignal.png}
%\caption{Example of aquired EEG signal from Emotiv device}
%\label{fig:eegsignal}
%\end{figure}

%At the begining we are going to focus on signals similar to those presented in Figure \ref{fig:eegsignal}.


%\bibliography{ksi}



\end{document}
